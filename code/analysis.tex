\documentclass[11pt,]{article}
\usepackage[left=1in,top=1in,right=1in,bottom=1in]{geometry}
\newcommand*{\authorfont}{\fontfamily{phv}\selectfont}
\usepackage[]{mathpazo}


  \usepackage[T1]{fontenc}
  \usepackage[utf8]{inputenc}




\usepackage{abstract}
\renewcommand{\abstractname}{}    % clear the title
\renewcommand{\absnamepos}{empty} % originally center

\renewenvironment{abstract}
 {{%
    \setlength{\leftmargin}{0mm}
    \setlength{\rightmargin}{\leftmargin}%
  }%
  \relax}
 {\endlist}

\makeatletter
\def\@maketitle{%
  \newpage
%  \null
%  \vskip 2em%
%  \begin{center}%
  \let \footnote \thanks
    {\fontsize{18}{20}\selectfont\raggedright  \setlength{\parindent}{0pt} \@title \par}%
}
%\fi
\makeatother




\setcounter{secnumdepth}{0}

\usepackage{color}
\usepackage{fancyvrb}
\newcommand{\VerbBar}{|}
\newcommand{\VERB}{\Verb[commandchars=\\\{\}]}
\DefineVerbatimEnvironment{Highlighting}{Verbatim}{commandchars=\\\{\}}
% Add ',fontsize=\small' for more characters per line
\usepackage{framed}
\definecolor{shadecolor}{RGB}{248,248,248}
\newenvironment{Shaded}{\begin{snugshade}}{\end{snugshade}}
\newcommand{\AlertTok}[1]{\textcolor[rgb]{0.94,0.16,0.16}{#1}}
\newcommand{\AnnotationTok}[1]{\textcolor[rgb]{0.56,0.35,0.01}{\textbf{\textit{#1}}}}
\newcommand{\AttributeTok}[1]{\textcolor[rgb]{0.77,0.63,0.00}{#1}}
\newcommand{\BaseNTok}[1]{\textcolor[rgb]{0.00,0.00,0.81}{#1}}
\newcommand{\BuiltInTok}[1]{#1}
\newcommand{\CharTok}[1]{\textcolor[rgb]{0.31,0.60,0.02}{#1}}
\newcommand{\CommentTok}[1]{\textcolor[rgb]{0.56,0.35,0.01}{\textit{#1}}}
\newcommand{\CommentVarTok}[1]{\textcolor[rgb]{0.56,0.35,0.01}{\textbf{\textit{#1}}}}
\newcommand{\ConstantTok}[1]{\textcolor[rgb]{0.00,0.00,0.00}{#1}}
\newcommand{\ControlFlowTok}[1]{\textcolor[rgb]{0.13,0.29,0.53}{\textbf{#1}}}
\newcommand{\DataTypeTok}[1]{\textcolor[rgb]{0.13,0.29,0.53}{#1}}
\newcommand{\DecValTok}[1]{\textcolor[rgb]{0.00,0.00,0.81}{#1}}
\newcommand{\DocumentationTok}[1]{\textcolor[rgb]{0.56,0.35,0.01}{\textbf{\textit{#1}}}}
\newcommand{\ErrorTok}[1]{\textcolor[rgb]{0.64,0.00,0.00}{\textbf{#1}}}
\newcommand{\ExtensionTok}[1]{#1}
\newcommand{\FloatTok}[1]{\textcolor[rgb]{0.00,0.00,0.81}{#1}}
\newcommand{\FunctionTok}[1]{\textcolor[rgb]{0.00,0.00,0.00}{#1}}
\newcommand{\ImportTok}[1]{#1}
\newcommand{\InformationTok}[1]{\textcolor[rgb]{0.56,0.35,0.01}{\textbf{\textit{#1}}}}
\newcommand{\KeywordTok}[1]{\textcolor[rgb]{0.13,0.29,0.53}{\textbf{#1}}}
\newcommand{\NormalTok}[1]{#1}
\newcommand{\OperatorTok}[1]{\textcolor[rgb]{0.81,0.36,0.00}{\textbf{#1}}}
\newcommand{\OtherTok}[1]{\textcolor[rgb]{0.56,0.35,0.01}{#1}}
\newcommand{\PreprocessorTok}[1]{\textcolor[rgb]{0.56,0.35,0.01}{\textit{#1}}}
\newcommand{\RegionMarkerTok}[1]{#1}
\newcommand{\SpecialCharTok}[1]{\textcolor[rgb]{0.00,0.00,0.00}{#1}}
\newcommand{\SpecialStringTok}[1]{\textcolor[rgb]{0.31,0.60,0.02}{#1}}
\newcommand{\StringTok}[1]{\textcolor[rgb]{0.31,0.60,0.02}{#1}}
\newcommand{\VariableTok}[1]{\textcolor[rgb]{0.00,0.00,0.00}{#1}}
\newcommand{\VerbatimStringTok}[1]{\textcolor[rgb]{0.31,0.60,0.02}{#1}}
\newcommand{\WarningTok}[1]{\textcolor[rgb]{0.56,0.35,0.01}{\textbf{\textit{#1}}}}



\title{Political Donor Polarization in Wisconsin \thanks{Code and data available at: github.com/Rdahlke}  }



\author{\Large Ross Dahlke\vspace{0.05in} \newline\normalsize\emph{}  }


\date{}

\usepackage{titlesec}

\titleformat*{\section}{\normalsize\bfseries}
\titleformat*{\subsection}{\normalsize\itshape}
\titleformat*{\subsubsection}{\normalsize\itshape}
\titleformat*{\paragraph}{\normalsize\itshape}
\titleformat*{\subparagraph}{\normalsize\itshape}





\newtheorem{hypothesis}{Hypothesis}
\usepackage{setspace}


% set default figure placement to htbp
\makeatletter
\def\fps@figure{htbp}
\makeatother


% move the hyperref stuff down here, after header-includes, to allow for - \usepackage{hyperref}

\makeatletter
\@ifpackageloaded{hyperref}{}{%
\ifxetex
  \PassOptionsToPackage{hyphens}{url}\usepackage[setpagesize=false, % page size defined by xetex
              unicode=false, % unicode breaks when used with xetex
              xetex]{hyperref}
\else
  \PassOptionsToPackage{hyphens}{url}\usepackage[draft,unicode=true]{hyperref}
\fi
}

\@ifpackageloaded{color}{
    \PassOptionsToPackage{usenames,dvipsnames}{color}
}{%
    \usepackage[usenames,dvipsnames]{color}
}
\makeatother
\hypersetup{breaklinks=true,
            bookmarks=true,
            pdfauthor={Ross Dahlke ()},
             pdfkeywords = {state politics, political donations, network analysis, polarization},  
            pdftitle={Political Donor Polarization in Wisconsin},
            colorlinks=true,
            citecolor=blue,
            urlcolor=blue,
            linkcolor=magenta,
            pdfborder={0 0 0}}
\urlstyle{same}  % don't use monospace font for urls

% Add an option for endnotes. -----


% add tightlist ----------
\providecommand{\tightlist}{%
\setlength{\itemsep}{0pt}\setlength{\parskip}{0pt}}

% add some other packages ----------

% \usepackage{multicol}
% This should regulate where figures float
% See: https://tex.stackexchange.com/questions/2275/keeping-tables-figures-close-to-where-they-are-mentioned
\usepackage[section]{placeins}


\begin{document}
	
% \pagenumbering{arabic}% resets `page` counter to 1 
%
% \maketitle

{% \usefont{T1}{pnc}{m}{n}
\setlength{\parindent}{0pt}
\thispagestyle{plain}
{\fontsize{18}{20}\selectfont\raggedright 
\maketitle  % title \par  

}

{
   \vskip 13.5pt\relax \normalsize\fontsize{11}{12} 
\textbf{\authorfont Ross Dahlke} \hskip 15pt \emph{\small }   

}

}








\begin{abstract}

    \hbox{\vrule height .2pt width 39.14pc}

    \vskip 8.5pt % \small 

\noindent Since the 2012 recall election of Governor Scott Walker, political
polarization in Wisconsin has become a major topic of research. Research
on Wisconsin's politics has shown that there is mass polarization in
Wisconsin's electorate. However, there is little research done on
polarization of political donors and the policy implications of donor
polarization. This study intends to fill this research gap by studying
networks of political donors in Wisconsin. I use data from the Wisconsin
Campaign Finance Information System, Wisconsin's official campaign
finance database, to create network analyses of donors to candidates for
Wisconsin's state-wide and state-legislative races. Through network
analysis, this study found that political donor networks have become
more polarized since the 2012 recall elections. This conclusion that
donor networks---the people who are funding the state's elections---have
recently become more polarized provides context for the mass
polarization of Wisconsin's electorate.


\vskip 8.5pt \noindent \emph{Keywords}: state politics, political donations, network analysis, polarization \par

    \hbox{\vrule height .2pt width 39.14pc}



\end{abstract}


\vskip -8.5pt


 % removetitleabstract

\noindent  

\hypertarget{introduction}{%
\section{Introduction}\label{introduction}}

The effect of money in politics has been heavily studied. Particularly,
studies have found a connection between politicians' polarization and
political donors' actions. The elimination of political donors
altogether through publicly funded elections resulted in more polarized
candidates because access-oriented interest groups acted as a moderating
force (Hall 2014).

\hypertarget{methodology}{%
\section{Methodology}\label{methodology}}

All data on political contributions came from the Wisconsin Campaign
Finance Information System (CFIS). I exported all contributions to State
Assembly, State Senate, and Gubernatorial races from the 2010, 2012, and
2014 elections. This dataset does not include donations to party
committees, although it does include disbursements from these
committees. I manually created a table of the parties of each of all the
campaigns receiving contributions in this timeframe and added the party
of the campaign receiving the donation to this dataset.

\begin{Shaded}
\begin{Highlighting}[]
\NormalTok{donations <-}\StringTok{ }\KeywordTok{readRDS}\NormalTok{(}\StringTok{"../data/wi_donations.RDA"}\NormalTok{) }\OperatorTok\StringTok{ }
\StringTok{  }\KeywordTok{mutate}\NormalTok{(}\DataTypeTok{source =} \KeywordTok{str_to_lower}\NormalTok{(}\KeywordTok{str_replace}\NormalTok{(source, }\StringTok{"[[:punct:]]"}\NormalTok{, }\StringTok{" "}\NormalTok{)),}
         \DataTypeTok{election_year =}  \KeywordTok{as.character}\NormalTok{(election_year)) }

\NormalTok{anon_donations_n <-}\StringTok{ }\NormalTok{donations }\OperatorTok\StringTok{ }
\StringTok{  }\KeywordTok{filter}\NormalTok{(}\KeywordTok{str_detect}\NormalTok{(source, }\KeywordTok{c}\NormalTok{(}\StringTok{"unitemized|anonymous"}\NormalTok{)) }\OperatorTok{==}\StringTok{ }\NormalTok{T) }\OperatorTok\StringTok{ }
\StringTok{  }\KeywordTok{count}\NormalTok{() }\OperatorTok\StringTok{ }
\StringTok{  }\KeywordTok{pull}\NormalTok{()}

\NormalTok{donations_}\DecValTok{2}\NormalTok{ <-}\StringTok{ }\NormalTok{donations }\OperatorTok\StringTok{ }
\StringTok{  }\KeywordTok{filter}\NormalTok{(}\KeywordTok{str_detect}\NormalTok{(source, }\KeywordTok{c}\NormalTok{(}\StringTok{"unitemized|anonymous"}\NormalTok{)) }\OperatorTok{==}\StringTok{ }\NormalTok{F) }\OperatorTok\StringTok{ }
\StringTok{  }\KeywordTok{mutate}\NormalTok{(}\DataTypeTok{zip_5 =} \KeywordTok{str_sub}\NormalTok{(zip, }\DecValTok{1}\NormalTok{, }\DecValTok{5}\NormalTok{),}
         \DataTypeTok{refined_source =}\NormalTok{ refinr}\OperatorTok{::}\KeywordTok{n_gram_merge}\NormalTok{(refinr}\OperatorTok{::}\KeywordTok{key_collision_merge}\NormalTok{(source)),}
         \DataTypeTok{refined_source_zip =} \KeywordTok{paste0}\NormalTok{(refined_source,}\StringTok{" : "}\NormalTok{,zip_}\DecValTok{5}\NormalTok{)) }

\NormalTok{filtered_donations <-}\StringTok{ }\NormalTok{donations_}\DecValTok{2} \OperatorTok\StringTok{ }
\StringTok{  }\KeywordTok{group_by}\NormalTok{(election_year, refined_source_zip) }\OperatorTok\StringTok{ }
\StringTok{  }\KeywordTok{mutate}\NormalTok{(}\DataTypeTok{source_count =} \KeywordTok{n}\NormalTok{()) }\OperatorTok\StringTok{ }
\StringTok{  }\KeywordTok{filter}\NormalTok{(source_count }\OperatorTok{>}\StringTok{ }\DecValTok{1}\NormalTok{) }\OperatorTok\StringTok{ }
\StringTok{  }\KeywordTok{group_by}\NormalTok{(election_year, target) }\OperatorTok\StringTok{ }
\StringTok{  }\KeywordTok{mutate}\NormalTok{(}\DataTypeTok{target_count =} \KeywordTok{n}\NormalTok{()) }\OperatorTok\StringTok{ }
\StringTok{  }\KeywordTok{filter}\NormalTok{(target_count }\OperatorTok{>}\StringTok{ }\DecValTok{20}\NormalTok{) }\OperatorTok\StringTok{ }
\StringTok{  }\KeywordTok{ungroup}\NormalTok{()}
\end{Highlighting}
\end{Shaded}

I started with \_\_\_ donations. To clean the data, I filtered out
unitemized/ anonymous donations, removed punctuation from the names of
the donors, and used Open Refine via the \texttt{refiner} R package to
standardize names (for example, Jim versus James). Next, I created a
unique identifier for donors by combining their standardized name with
their zip code. This identifier was created to be able to link donors
who contributed across multiple campaigns in multiple years without
considering two different people, with the same name, from different
locations to be the same person.

Next, I derived the partisanship of each donor in each election cycle. I
calculated each donor's partisanship by taking the percent of donations
that each donor gave to Republicans divided by their donations to
Republicans and Democrats. I took that ``percent donated to
Republicans'' and rescaled it from -1 to 1, where -1 represents the most
Democratic donors, and 1 the most Republican donors. I also calculated
each individual's party bin: if more than 75\% of donations were to
Democrats, they were labeled as a Democrat; if more than 75\% of
donations were to Republicans, they were labeled as a Republican; if
their donations were somewhere inbetween, they were labeled as being a
bipartisan donor.

\begin{Shaded}
\begin{Highlighting}[]
\NormalTok{by_donor <-}\StringTok{ }\NormalTok{filtered_donations }\OperatorTok\StringTok{ }
\StringTok{  }\KeywordTok{filter}\NormalTok{(party }\OperatorTok{!=}\StringTok{ "other"}\NormalTok{) }\OperatorTok\StringTok{ }
\StringTok{  }\KeywordTok{group_by}\NormalTok{(election_year, refined_source_zip, party) }\OperatorTok\StringTok{ }
\StringTok{  }\KeywordTok{summarize}\NormalTok{(}\DataTypeTok{contribution =} \KeywordTok{sum}\NormalTok{(contribution)) }\OperatorTok\StringTok{ }
\StringTok{  }\KeywordTok{ungroup}\NormalTok{() }\OperatorTok\StringTok{ }
\StringTok{  }\KeywordTok{pivot_wider}\NormalTok{(}\DataTypeTok{names_from =}\NormalTok{ party, }
              \DataTypeTok{values_from =}\NormalTok{ contribution) }\OperatorTok\StringTok{ }
\StringTok{  }\KeywordTok{mutate}\NormalTok{(}\DataTypeTok{rep =} \KeywordTok{replace_na}\NormalTok{(rep, }\DecValTok{0}\NormalTok{),}
         \DataTypeTok{dem =} \KeywordTok{replace_na}\NormalTok{(dem, }\DecValTok{0}\NormalTok{),}
         \DataTypeTok{total_contributions =}\NormalTok{ rep }\OperatorTok{+}\StringTok{ }\NormalTok{dem,}
         \DataTypeTok{per_rep =}\NormalTok{ rep }\OperatorTok{/}\StringTok{ }\NormalTok{total_contributions,}
         \DataTypeTok{partisanship =}\NormalTok{ scales}\OperatorTok{::}\KeywordTok{rescale}\NormalTok{(per_rep, }\DataTypeTok{to =} \KeywordTok{c}\NormalTok{(}\OperatorTok{-}\DecValTok{1}\NormalTok{, }\DecValTok{1}\NormalTok{)),}
         \DataTypeTok{abs_partisanship =} \KeywordTok{abs}\NormalTok{(partisanship),}
         \DataTypeTok{party_bin =} \KeywordTok{case_when}\NormalTok{(}
\NormalTok{           per_rep }\OperatorTok{<}\StringTok{ }\FloatTok{.25} \OperatorTok{~}\StringTok{ "democrat"}\NormalTok{,}
\NormalTok{           per_rep }\OperatorTok{>=}\StringTok{ }\FloatTok{.25} \OperatorTok{&}\StringTok{ }\NormalTok{per_rep }\OperatorTok{<=}\StringTok{ }\FloatTok{.75} \OperatorTok{~}\StringTok{ "bipartisan"}\NormalTok{,}
\NormalTok{           per_rep }\OperatorTok{>}\StringTok{ }\FloatTok{.75} \OperatorTok{~}\StringTok{ "republican"}
\NormalTok{         ))}
\end{Highlighting}
\end{Shaded}

\begin{verbatim}
## `summarise()` regrouping output by 'election_year', 'refined_source_zip' (override with `.groups` argument)
\end{verbatim}

To quantify the levels of polarization in each election cycle, I
calculated two statistics: network modularity and average absolute
partisanship of donors.

First, political donations can be thought of as a network where donors
and candidates are nodes and donations connecting donors and candidates
are edges. This conceptualization of the political donor landscape as
network allows us to examine the network structure and calculate network
statistics on the graph of donors and candidates. One of the most useful
network statistics for measuring polarization in a network's modularity.

The modularity of a graph measures how good the division of groups (such
as political parties) is by calculating ``the number of edges falling
within groups minus the expected number in an equivalent network with
edges placed at random''
(\url{https://www.pnas.org/content/103/23/8577}). The modularity of a
network falls in range {[}need latex{]} {[}-1/2, 1{]}. If the modularity
is positive, the number of edges that remain within each group is
greater than the expected number to remain in-group based on chance. The
higher the modularity, the greater the concentration of edges within
each groups. In other words, the higher the modularity of a network, the
higher the polarization among the groups.

I calculated the modularity of the network graphs of each election cycle
(2010, 2012, 2014). I used candidates' declared parties and donors'
party bin as the groups for the modularity calculation. The modularity
of the network graph of each election is in Table 1.

\begin{Shaded}
\begin{Highlighting}[]
\NormalTok{modularity_calc <-}\StringTok{ }\ControlFlowTok{function}\NormalTok{(}\DataTypeTok{year =}\NormalTok{ year)\{}
\NormalTok{  nodes_w_party <-}\StringTok{ }\NormalTok{by_donor }\OperatorTok\StringTok{ }
\StringTok{  }\KeywordTok{filter}\NormalTok{(election_year }\OperatorTok{==}\StringTok{ }\NormalTok{year) }\OperatorTok\StringTok{ }
\StringTok{  }\KeywordTok{select}\NormalTok{(refined_source_zip, party_bin) }\OperatorTok\StringTok{ }
\StringTok{  }\KeywordTok{rename}\NormalTok{(}\DataTypeTok{node =}\NormalTok{ refined_source_zip,}
         \DataTypeTok{party =}\NormalTok{ party_bin) }\OperatorTok\StringTok{ }
\StringTok{  }\KeywordTok{rbind}\NormalTok{(donations_}\DecValTok{2} \OperatorTok\StringTok{ }
\StringTok{  }\KeywordTok{distinct}\NormalTok{(target, party) }\OperatorTok\StringTok{ }
\StringTok{  }\KeywordTok{filter}\NormalTok{(party }\OperatorTok{!=}\StringTok{ "other"}\NormalTok{) }\OperatorTok\StringTok{ }
\StringTok{    }\KeywordTok{rename}\NormalTok{(}\DataTypeTok{node =}\NormalTok{ target) }\OperatorTok\StringTok{ }
\StringTok{    }\KeywordTok{mutate}\NormalTok{(}\DataTypeTok{party  =} \KeywordTok{case_when}\NormalTok{(}
\NormalTok{      party }\OperatorTok{==}\StringTok{ "rep"} \OperatorTok{~}\StringTok{ "republican"}\NormalTok{,}
\NormalTok{      party }\OperatorTok{==}\StringTok{ "dem"} \OperatorTok{~}\StringTok{ "democrat"}
\NormalTok{    ))) }\OperatorTok\StringTok{ }
\StringTok{  }\KeywordTok{distinct}\NormalTok{(node, party) }\OperatorTok\StringTok{ }
\StringTok{  }\KeywordTok{mutate}\NormalTok{(}\DataTypeTok{party_num =} \KeywordTok{case_when}\NormalTok{(}
\NormalTok{    party }\OperatorTok{==}\StringTok{ "democrat"} \OperatorTok{~}\StringTok{ }\DecValTok{1}\NormalTok{,}
\NormalTok{    party }\OperatorTok{==}\StringTok{ "bipartisan"} \OperatorTok{~}\StringTok{ }\DecValTok{2}\NormalTok{,}
\NormalTok{    party }\OperatorTok{==}\StringTok{ "republican"} \OperatorTok{~}\StringTok{ }\DecValTok{3}
\NormalTok{  ))}
  

\NormalTok{filtered_donations }\OperatorTok
\StringTok{  }\KeywordTok{filter}\NormalTok{(election_year }\OperatorTok{==}\StringTok{ }\NormalTok{year }\OperatorTok{&}\StringTok{ }\NormalTok{party }\OperatorTok{!=}\StringTok{ "other"}\NormalTok{) }\OperatorTok\StringTok{ }
\StringTok{  }\KeywordTok{select}\NormalTok{(refined_source_zip, target, contribution) }\OperatorTok\StringTok{ }
\StringTok{  }\KeywordTok{graph_from_data_frame}\NormalTok{(}\DataTypeTok{vertices =}\NormalTok{ nodes_w_party) }\OperatorTok\StringTok{ }
\StringTok{  }\KeywordTok{modularity}\NormalTok{(nodes_w_party}\OperatorTok{$}\NormalTok{party_num)}
\NormalTok{\}}
\end{Highlighting}
\end{Shaded}

\begin{Shaded}
\begin{Highlighting}[]
\NormalTok{election_years <-}\StringTok{ }\NormalTok{by_donor }\OperatorTok\StringTok{ }
\StringTok{  }\KeywordTok{distinct}\NormalTok{(election_year) }\OperatorTok\StringTok{ }
\StringTok{  }\KeywordTok{pull}\NormalTok{()}
\end{Highlighting}
\end{Shaded}

\begin{Shaded}
\begin{Highlighting}[]
\KeywordTok{tibble}\NormalTok{(election_years) }\OperatorTok\StringTok{ }
\StringTok{  }\KeywordTok{group_by}\NormalTok{(election_years) }\OperatorTok\StringTok{ }
\StringTok{  }\KeywordTok{mutate}\NormalTok{(}\DataTypeTok{modularity =} \KeywordTok{modularity_calc}\NormalTok{(election_years))}
\end{Highlighting}
\end{Shaded}

\begin{verbatim}
## # A tibble: 3 x 2
## # Groups:   election_years [3]
##   election_years modularity
##   <chr>               <dbl>
## 1 2010                0.399
## 2 2012                0.491
## 3 2014                0.480
\end{verbatim}

In addition to calculating the change in modularity of each of the
election cycles, I also analyzed the change in mean absolute
partisanship of the donors in each election cycle.

I defined a donor's absolute partisanship as the absolute value of their
partisanship score (which is on a scale from -1 to 1). Therefore, the
larger a donor's absolute the partisanship, the higher percentage of
their money that they contributed to a single party. To calculate the
significance in the difference of the mean absolute partisanship, I use
a bootstrap methodology with 1000 replications. The results of the
bootstrap are found in Table 2.

\begin{Shaded}
\begin{Highlighting}[]
\NormalTok{diff_boostrap <-}\StringTok{ }\ControlFlowTok{function}\NormalTok{(}\DataTypeTok{year_1 =} \StringTok{"2014"}\NormalTok{,}
                          \DataTypeTok{year_2 =} \StringTok{"2012"}\NormalTok{,}
                          \DataTypeTok{replications =} \DecValTok{100}\NormalTok{) \{}
\NormalTok{  year_}\DecValTok{1}\NormalTok{ <-}\StringTok{ }\KeywordTok{as.character}\NormalTok{(year_}\DecValTok{1}\NormalTok{)}
\NormalTok{  year_}\DecValTok{2}\NormalTok{ <-}\StringTok{ }\KeywordTok{as.character}\NormalTok{(year_}\DecValTok{2}\NormalTok{)}
  
\NormalTok{  bootstrap <-}\StringTok{ }\NormalTok{by_donor }\OperatorTok\StringTok{ }
\StringTok{    }\KeywordTok{filter}\NormalTok{(election_year }\OperatorTok\StringTok{ }\KeywordTok{c}\NormalTok{(year_}\DecValTok{1}\NormalTok{, year_}\DecValTok{2}\NormalTok{)) }\OperatorTok\StringTok{ }
\StringTok{    }\KeywordTok{specify}\NormalTok{(abs_partisanship }\OperatorTok{~}\StringTok{ }\NormalTok{election_year) }\OperatorTok\StringTok{ }
\StringTok{    }\KeywordTok{generate}\NormalTok{(}\DataTypeTok{reps =}\NormalTok{ replications, }\DataTypeTok{type =} \StringTok{"bootstrap"}\NormalTok{) }\OperatorTok\StringTok{ }
\StringTok{    }\KeywordTok{calculate}\NormalTok{(}\DataTypeTok{stat =} \StringTok{"diff in means"}\NormalTok{, }\DataTypeTok{order =} \KeywordTok{c}\NormalTok{(year_}\DecValTok{1}\NormalTok{, year_}\DecValTok{2}\NormalTok{))}
  

\NormalTok{  bootstrap }\OperatorTok\StringTok{ }
\StringTok{    }\KeywordTok{get_ci}\NormalTok{() }\OperatorTok\StringTok{ }
\StringTok{    }\KeywordTok{cbind}\NormalTok{(bootstrap }\OperatorTok\StringTok{ }
\StringTok{            }\KeywordTok{summarize}\NormalTok{(}\DataTypeTok{mean_diff =} \KeywordTok{mean}\NormalTok{(stat))) }\OperatorTok\StringTok{ }
\StringTok{    }\KeywordTok{cbind}\NormalTok{(bootstrap }\OperatorTok\StringTok{ }
\StringTok{            }\KeywordTok{get_p_value}\NormalTok{(}\DataTypeTok{obs_stat =} \DecValTok{0}\NormalTok{, }\DataTypeTok{direction =} \StringTok{"two_sided"}\NormalTok{))}
  

\NormalTok{                          \}}
\end{Highlighting}
\end{Shaded}

\begin{Shaded}
\begin{Highlighting}[]
\NormalTok{election_years_expanded <-}\StringTok{ }\KeywordTok{expand.grid}\NormalTok{(}\DataTypeTok{year1 =} \KeywordTok{as.character}\NormalTok{(election_years),}
            \DataTypeTok{year2 =} \KeywordTok{as.character}\NormalTok{(election_years)) }\OperatorTok\StringTok{ }
\StringTok{  }\KeywordTok{filter}\NormalTok{(year1 }\OperatorTok{!=}\StringTok{ }\NormalTok{year2 }\OperatorTok{&}\StringTok{ }
\StringTok{           }\KeywordTok{as.numeric}\NormalTok{(year1) }\OperatorTok{>}\StringTok{ }\KeywordTok{as.numeric}\NormalTok{(year2) }\OperatorTok{&}\StringTok{ }
\StringTok{           }\NormalTok{(}\KeywordTok{as.numeric}\NormalTok{(year1) }\OperatorTok{-}\StringTok{ }\KeywordTok{as.numeric}\NormalTok{(year2) }\OperatorTok{<}\StringTok{ }\DecValTok{2}\NormalTok{)) }\OperatorTok\StringTok{ }
\StringTok{  }\KeywordTok{ungroup}\NormalTok{() }

\NormalTok{partisanship_diff_results <-}\StringTok{ }\KeywordTok{cbind}\NormalTok{(election_years_expanded, }
                                   \KeywordTok{Map}\NormalTok{(diff_boostrap, }
\NormalTok{                                       election_years_expanded}\OperatorTok{$}\NormalTok{year1, }
\NormalTok{                                       election_years_expanded}\OperatorTok{$}\NormalTok{year2, }
                                       \DecValTok{1000}\NormalTok{) }\OperatorTok\StringTok{ }
\StringTok{                                     }\KeywordTok{tibble}\NormalTok{() }\OperatorTok\StringTok{ }
\StringTok{                                     }\KeywordTok{unnest}\NormalTok{())}
\end{Highlighting}
\end{Shaded}

\begin{verbatim}
## Warning: `cols` is now required.
## Please use `cols = c(.)`
\end{verbatim}

\hypertarget{references}{%
\section*{References}\label{references}}
\addcontentsline{toc}{section}{References}

\hypertarget{refs}{}
\leavevmode\hypertarget{ref-hall2014}{}%
Hall, Andrew B. 2014. ``How the Public Funding of Elections Increases
Candidate Polarization.''





\newpage
\singlespacing 
\end{document}
