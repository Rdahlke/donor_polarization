\documentclass[11pt,]{article}
\usepackage[left=1in,top=1in,right=1in,bottom=1in]{geometry}
\newcommand*{\authorfont}{\fontfamily{phv}\selectfont}
\usepackage[]{mathpazo}


  \usepackage[T1]{fontenc}
  \usepackage[utf8]{inputenc}




\usepackage{abstract}
\renewcommand{\abstractname}{}    % clear the title
\renewcommand{\absnamepos}{empty} % originally center

\renewenvironment{abstract}
 {{%
    \setlength{\leftmargin}{0mm}
    \setlength{\rightmargin}{\leftmargin}%
  }%
  \relax}
 {\endlist}

\makeatletter
\def\@maketitle{%
  \newpage
%  \null
%  \vskip 2em%
%  \begin{center}%
  \let \footnote \thanks
    {\fontsize{18}{20}\selectfont\raggedright  \setlength{\parindent}{0pt} \@title \par}%
}
%\fi
\makeatother




\setcounter{secnumdepth}{0}




\title{Political Donor Polarization in Wisconsin \thanks{Code and data available at: github.com/rossdahlke}  }



\author{\Large Ross Dahlke\vspace{0.05in} \newline\normalsize\emph{}  }


\date{}

\usepackage{titlesec}

\titleformat*{\section}{\normalsize\bfseries}
\titleformat*{\subsection}{\normalsize\itshape}
\titleformat*{\subsubsection}{\normalsize\itshape}
\titleformat*{\paragraph}{\normalsize\itshape}
\titleformat*{\subparagraph}{\normalsize\itshape}





\newtheorem{hypothesis}{Hypothesis}
\usepackage{setspace}


% set default figure placement to htbp
\makeatletter
\def\fps@figure{htbp}
\makeatother


% move the hyperref stuff down here, after header-includes, to allow for - \usepackage{hyperref}

\makeatletter
\@ifpackageloaded{hyperref}{}{%
\ifxetex
  \PassOptionsToPackage{hyphens}{url}\usepackage[setpagesize=false, % page size defined by xetex
              unicode=false, % unicode breaks when used with xetex
              xetex]{hyperref}
\else
  \PassOptionsToPackage{hyphens}{url}\usepackage[draft,unicode=true]{hyperref}
\fi
}

\@ifpackageloaded{color}{
    \PassOptionsToPackage{usenames,dvipsnames}{color}
}{%
    \usepackage[usenames,dvipsnames]{color}
}
\makeatother
\hypersetup{breaklinks=true,
            bookmarks=true,
            pdfauthor={Ross Dahlke ()},
             pdfkeywords = {state politics, political donations, network analysis, polarization},  
            pdftitle={Political Donor Polarization in Wisconsin},
            colorlinks=true,
            citecolor=blue,
            urlcolor=blue,
            linkcolor=magenta,
            pdfborder={0 0 0}}
\urlstyle{same}  % don't use monospace font for urls

% Add an option for endnotes. -----


% add tightlist ----------
\providecommand{\tightlist}{%
\setlength{\itemsep}{0pt}\setlength{\parskip}{0pt}}

% add some other packages ----------

% \usepackage{multicol}
% This should regulate where figures float
% See: https://tex.stackexchange.com/questions/2275/keeping-tables-figures-close-to-where-they-are-mentioned
\usepackage[section]{placeins}


\begin{document}
	
% \pagenumbering{arabic}% resets `page` counter to 1 
%
% \maketitle

{% \usefont{T1}{pnc}{m}{n}
\setlength{\parindent}{0pt}
\thispagestyle{plain}
{\fontsize{18}{20}\selectfont\raggedright 
\maketitle  % title \par  

}

{
   \vskip 13.5pt\relax \normalsize\fontsize{11}{12} 
\textbf{\authorfont Ross Dahlke} \hskip 15pt \emph{\small }   

}

}








\begin{abstract}

    \hbox{\vrule height .2pt width 39.14pc}

    \vskip 8.5pt % \small 

\noindent Since the 2012 recall election of Governor Scott Walker, political
polarization in Wisconsin has become a major topic in Wisconin's
politics. Mass polarization among Wisconsin's electorate is well
documented. However, there is little research done on polarization of
political donors and the policy implications of donor polarization. This
study intends to fill this research gap by studying networks of
political donors in Wisconsin. I use data from the Wisconsin Campaign
Finance Information System, Wisconsin's official campaign finance
database, to create networks of donors to candidates for Wisconsin's
state-wide and state-legislative races. By using community modularity, I
empirically show that Republican and Democratic donor networks polarized
in the 2012 election cycle during the recall election of Scott Walker
and remained polarized in the 2014 election cycle in which Governor
Walker was reelected. I also validate these results using a
non-parametric bootrapping method. I discuss the implications of this
conclusion \_\_\_\_\_\_


\vskip 8.5pt \noindent \emph{Keywords}: state politics, political donations, network analysis, polarization \par

    \hbox{\vrule height .2pt width 39.14pc}



\end{abstract}


\vskip -8.5pt


 % removetitleabstract

\noindent \doublespacing 

Political campaign finance plays an important role in the American
political system. This significance is evidenced by the attention that
academic researchers pay the topic as well as the many different
contexts in which campaign finance is studied. For example, academics
have researched the impact of political donations on roll call voting in
the U.S. Congress (Douglas D. Roscoe 2005; Stratmann 1991), gender
representation in political parties (Melody Crowder-Meyer 2018; Barber
2016; Karin E. Kitchens 2016; Danielle M. Thomsen 2017), ability to win
political campaigns (Bonica 2017; Bonneau 2007), the connection between
money raised and public attention (William Curtis Ellis 2017), judicial
function (Vernon Valentine Palmer 2008), perceptions of corruption
(Shaun Bowler 2015), political economy and stock returns (Akey 2015;
Anthony Fowler 2020; Michael J. Cooper 2010), and the significant amount
of time that candidates and legislators devote to fundraising
({\textbf{???}}).

Similarities to other political actors? Same psychological processes?

Are political donors being becoming polarized like other political
actors?

Some research has started to investigate the connection between donors
and candidates as ``an important part of the story of the polarization
of American politics'' (Barber 2016). Barber found that higher
state-level contribution limits for individuals leads to more moderate
legislator in that state. And higher limits for political action
committees (PACs) results in more moderate candidates.

While polling may show the polarization of political donors, there has
not been much research into actual actions of political donors that show
evidence for their polarization

Voters have more polarized voting, office holders can have more
polarized voting records.

This study asks the question: Like other political actors, is there
evidence of political donors becoming polarized?

Hypothesis: Yes, they are becoming more polarized, in part because they
are already a largely polarized group. We would expect more polarization

Since donors are believed to play an out-sized role in our democracy,
understanding whether political donors have become more polarized is
important for identifying both the causes and effects of polarization in
our entire political system

\hypertarget{methodology}{%
\section{Methodology}\label{methodology}}

All data on political contributions came from the Wisconsin Campaign
Finance Information System (CFIS). I exported all contributions to State
Assembly, State Senate, and Gubernatorial races from the 2010, 2012, and
2014 elections. This dataset does not include donations to party
committees, although it does include disbursements from these
committees. I manually created a table of the parties of each of all the
campaigns receiving contributions in this timeframe and added the party
of the campaign receiving the donation to this dataset.

I started with \_\_\_ donations. To clean the data, I filtered out
unitemized/ anonymous donations, removed punctuation from the names of
the donors, and used Open Refine via the \texttt{refiner} R package to
standardize names (for example, Jim versus James). Next, I created a
unique identifier for donors by combining their standardized name with
their zip code. This identifier was created to be able to link donors
who contributed across multiple campaigns in multiple years without
considering two different people, with the same name, from different
locations to be the same person.

Next, I derived the partisanship of each donor in each election cycle. I
calculated each donor's partisanship by taking the percent of donations
that each donor gave to Republicans divided by their donations to
Republicans and Democrats. I took that ``percent donated to
Republicans'' and rescaled it from -1 to 1, where -1 represents the most
Democratic donors, and 1 the most Republican donors. I also calculated
each individual's party bin: if more than 75\% of donations were to
Democrats, they were labeled as a Democrat; if more than 75\% of
donations were to Republicans, they were labeled as a Republican; if
their donations were somewhere inbetween, they were labeled as being a
bipartisan donor.

To quantify the levels of polarization in each election cycle, I
calculated two statistics: network modularity and average absolute
partisanship of donors.

First, political donations can be thought of as a network where donors
and candidates are nodes and donations connecting donors and candidates
are edges. This conceptualization of the political donor landscape as
network allows us to examine the network structure and calculate network
statistics on the graph of donors and candidates. One of the most useful
network statistics for measuring polarization in a network's modularity.

The modularity of a graph measures how good the division of groups (such
as political parties) is by calculating ``the number of edges falling
within groups minus the expected number in an equivalent network with
edges placed at random''
(\url{https://www.pnas.org/content/103/23/8577}). The modularity of a
network falls in range {[}need latex{]} {[}-1/2, 1{]}. If the modularity
is positive, the number of edges that remain within each group is
greater than the expected number to remain in-group based on chance. The
higher the modularity, the greater the concentration of edges within
each groups. In other words, the higher the modularity of a network, the
higher the polarization among the groups.

I calculated the modularity of the network graphs of each election cycle
(2010, 2012, 2014). I used candidates' declared parties and donors'
party bin as the groups for the modularity calculation. The modularity
of the network graph of each election is in Table 1.

In addition to calculating the change in modularity of each of the
election cycles, I also analyzed the change in mean absolute
partisanship of the donors in each election cycle.

I defined a donor's absolute partisanship as the absolute value of their
partisanship score (which is on a scale from -1 to 1). Therefore, the
larger a donor's absolute the partisanship, the higher percentage of
their money that they contributed to a single party. To calculate the
significance in the difference of the mean absolute partisanship, I use
a bootstrap methodology with 1000 replications. The results of the
bootstrap are found in Table 2.

\hypertarget{references}{%
\section*{References}\label{references}}
\addcontentsline{toc}{section}{References}

\hypertarget{refs}{}
\leavevmode\hypertarget{ref-akey2015}{}%
Akey, Pat. 2015. ``Valuing Changes in Political Networks: Evidence from
Campaign Contributions to Close Congressional Elections.'' \emph{The
Review of Financial Studies} 28 (11): 3188--3223.

\leavevmode\hypertarget{ref-garro2020}{}%
Anthony Fowler, Jörg L. Spenkuch, Haritz Garro. 2020. ``Quid Pro Quo?
Corporate Returns to Campaign Contributions.'' \emph{The Journal of
Politics} 82 (3): 844--58.

\leavevmode\hypertarget{ref-barber2016}{}%
Barber, Michael J. 2016. ``Ideological Donors, Contribution Limits, and
the Polarization of American Legislatures.'' \emph{The Journal of
Politics} 78 (1): 296--310.

\leavevmode\hypertarget{ref-bonica2017}{}%
Bonica, Adam. 2017. ``Professional Networks, Early Fundraising, and
Electoral Success.'' \emph{Election Law Journal: Rules, Politics, and
Policy} 16 (1): 153--71.

\leavevmode\hypertarget{ref-bonneau2007}{}%
Bonneau, Chris W. 2007. ``Campaign Fundraising in State Supreme Court
Elections.'' \emph{Social Science Quartlery} 88 (1): 68--85.

\leavevmode\hypertarget{ref-thomsen2017}{}%
Danielle M. Thomsen, Michele L. Swers. 2017. ``Which Women Can Run?
Gender, Partisanship, and Candidate Donor Networks.'' \emph{Political
Research Quarterly} 70 (2): 449--63.

\leavevmode\hypertarget{ref-roscoe2005}{}%
Douglas D. Roscoe, Shannon Jenkins. 2005. ``A Meta-Analysis of Campaign
Contributions' Impact on Roll Call Voting.'' \emph{Social Science
Quartlery} 86 (1): 52--68.

\leavevmode\hypertarget{ref-kitchens2016}{}%
Karin E. Kitchens, Michele L. Swers. 2016. ``Why Aren't There More
Republican Women in Congress? Gender, Partisanship, and Fundraising
Support in the 2010 and 2012 Elections.'' \emph{Politics \& Gender} 12
(4): 648--76.

\leavevmode\hypertarget{ref-crowder-meyer2018}{}%
Melody Crowder-Meyer, Rosalyn Cooperman. 2018. ``Can't Buy Them Love:
How Party Culture Among Donors Contributes to the Party Gap in Women's
Representation.'' \emph{The Journal of Politics} 80 (4): 1211--24.

\leavevmode\hypertarget{ref-cooper2010}{}%
Michael J. Cooper, Alexei V. Ovtchinnikov, Huseyin Gulen. 2010.
``Corporate Political Contributions and Stock Returns.'' \emph{The
Journal of Finance} 65 (2): 687--724.

\leavevmode\hypertarget{ref-bowler2015}{}%
Shaun Bowler, Todd Donovan. 2015. ``Campaign Money, Congress, and
Perceptions of Corruption.'' \emph{American Politics Research} 44 (2):
272--95.

\leavevmode\hypertarget{ref-stratmann1991}{}%
Stratmann, Thomas. 1991. ``What Do Campaign Contributions Buy?
Deciphering Causal Effects of Money and Votes.'' \emph{Southern Economic
Journal} 57 (3): 606--20.

\leavevmode\hypertarget{ref-palmer2008}{}%
Vernon Valentine Palmer, John Levendis. 2008. ``The Louisiana Supreme
Court in Question: An Empirical and Statistical Study of the Effects of
Money on the Judicial Function.'' \emph{Tulane Law Review} 82 (4):
1291--1314.

\leavevmode\hypertarget{ref-ellis2017}{}%
William Curtis Ellis, Colin Swearingen, Joseph T. Ripberger. 2017.
``Public Attention and Head-to-Head Campaign Fundraising: An Examination
of U.s. Senate Elections.'' \emph{American Review of Politics} 36 (1):
30--53.





\newpage
\singlespacing 
\end{document}
